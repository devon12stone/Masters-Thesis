\chapter{Introduction} 
\label{Chapter1}
%---------------------------------------------------------------------------------------
%	SECTION 1
%---------------------------------------------------------------------------------------

\section{Problem Description}

Over 1.7 billion adults around the world do not have access to basic financial services, even more, do not have access to a source of safe credit. Mobile banking platforms have been a driving force in worldwide financial inclusion. Since 2011, over 1 billion adults have gained access to a bank account for the first time \parencite{WorldBank}. Despite the rapid developments in financial inclusion, providing credit to the recently banked population remains an issue. \\

The recently banked population does not have a financial history and are required to develop their finical history with an institution before they can be deemed creditworthy. This can often be a time-consuming process and can cause financial strain. This problem is most commonly occurs within the recently banked population within developing countries. However, the issue does relate to young adults entering the financial market within developed countries. \\

Micro-Finance companies and larger corporate institutions are starting to provide solutions to this issue. The solution is being derived through the use of alternative data in conjunction with machine learning algorithms. Alternative data is being used to develop the models that drive credit scoring systems which grant or deny credit to consumers \parencite{BigDataMicroFiance}. 

%---------------------------------------------------------------------------------------
%	SECTION 2
%---------------------------------------------------------------------------------------

\section{Background}

Credit scoring is the set of modelling and decision techniques associated with autonomously adjudicating whether or not a potential borrower should be granted credit \parencite{PerceptronScoring}. The techniques involved are used to drive the strategies for determining the amount of credit a borrower  should receive, the period of repayment, and the interest rate due on the amount borrowed \parencite{CreditRiskSummary}. \\

Credit scoring systems range in scale from the rating of countries and global international companies to rating personal credits. The systems measure a potential borrower 's ability to repay a financial obligation. The systems do not forecast loan profitability. Rather, they are used to reduce credit risk and limit the number of loans that are not repaid, which in turn increases profitability \parencite{EarlyNNScoring}. \newpage

Traditional credit scoring involves considering a borrower's previous loan history when determining their credit score. Loan history data can comprise of loans that were taken from the institution granting the credit, or can be acquired from external credit bureaus. A consumers' previous loan performance directly influences the credit score they receive. If a consumer did not fully repay a previous loan, credit scoring systems will take this into consideration and assign the consumer a lower score \parencite{DynamicBehaviouralScoring}. \\

Since the beginning of the big data era, financial institutions have been able to access larger and larger volumes of data. Beyond the volume of data, financial intuitions have been able to access various types of data, from varying resources. Companies have been able to extract data from clients' short message service history, their call history, and data from clients' social media platforms. Data required from resources such as the ones mentioned is referred to as alternative data.\\

Credit scoring is one of the oldest applications of data analytics. Prior to the rise of machine learning, the traditional models used to drive credit scoring systems were limited to less complex techniques such as logistic regression, linear discriminant analysis and naive Bayes classifiers.\\

Since the rise of big data and machine learning, financial institutions have been able to train and use non-parametric models such as decision trees, support vector machines and neural networks to drive their credit decision systems \parencite{IntroToCreditModelling}. \\

The use of modern machine learning algorithms within credit scoring systems has not been fully supported within the financial industry. Machine learning algorithms are often complex and the predictions they produce can be difficult to explain. \\

Beyond their complexity, modern machine learning models dynamically evolve and are required to be regularly retrained on different data flows. Tracing the evolution of machine learning models and the data flows they are retrained on poses major issues for financial regulators \parencite{Regulation}. \\

%---------------------------------------------------------------------------------------
%	SECTION 3
%---------------------------------------------------------------------------------------

\section{Aims of Research}

Despite recent advancements in the use of alternative data in credit granting models \parencite{BigDataMicroFiance}, the research has been limited to using contact data on potential borrowers cellular device and network analysis techniques. This minor dissertation (m.d.) assesses the impact of using features generated from Short Messaging Service (SMS) data, only messages received from banking intuitions, and Android application data on borrowers` devices in credit scoring modelling. \\

Furthermore, the research investigates the benefits of using alternative data sources in terms of financial inclusion within third world countries. The research investigates if there is a significant lift in model performance when traditional credit data sources, such as sociodemographic and credit bureau data, are augmented with alternative data sources.  \\

Finally, the research investigates the performance of three machine learning techniques when used in credit scoring.  


%---------------------------------------------------------------------------------------
%	SECTION 4
%---------------------------------------------------------------------------------------
\section{Layout of the Paper}

Chapter \ref{Chapter2} presents a review of literature related to assigning first-time loan customers with a credit score and how a credit score relates to loan default prediction. Particular attention is given to the modelling techniques that have been used to drive credit scoring, and how those techniques have developed over time. Finally the chapter explores the various alternative data sources used in credit scoring and the techniques applied to make use of the sources. \\

Chapter \ref{Chapter3} presents the data wrangling, data processing and feature engineering completed in order to generate the training and testing sets used to develop the models produced for this m.d. The chapter thoroughly outlines the various processes taken to generate the alternative features. \\

Chapter \ref{Chapter4} displays and describes the various models developed. The chapter further describes the techniques behind each model and the feature sets used to train and test each model. \\

Chapter \ref{Chapter5} displays the results of each model developed and highlights the significant differences between each technique and  feature set used. \\

Chapter \ref{Chapter6} presents the findings and conclusions of the project. 



\chapter{Introduction} 
\label{Chapter1}
%---------------------------------------------------------------------------------------
%	SECTION 1
%---------------------------------------------------------------------------------------

\section{Problem Description}

More than 1.7 billion adults around the world do not have access to basic financial services; even more do not have access to a source of safe credit. Mobile banking platforms have been a driving force in worldwide financial inclusion. Since 2011, more than 1 billion adults have gained access to a bank account for the first time \parencite{WorldBank}. Despite the rapid developments in financial inclusion, providing credit to the recently banked \footnote{People that have recently gained access to a bank account.} population remains an issue. \\

The recently banked population does not have a financial history and are required to develop their financial history with an institution before they can be deemed creditworthy. This can often be a time-consuming process and can cause financial strain. This problem most occurs within the recently banked population in developing countries. However, the issue does relate to young adults entering the financial market within developed countries. \\

Micro-Finance companies and larger corporate institutions are starting to provide solutions to this issue. The solution is being derived through the use of alternative data in conjunction with machine learning algorithms. Alternative data - which includes data sourced from an individuals personal cellular device such as call and sms data, contact information, social media and other application data etc - is being used to develop the models that drive credit scoring systems, which grant or deny credit to consumers \parencite{BigDataMicroFiance}. 

%---------------------------------------------------------------------------------------
%	SECTION 2
%---------------------------------------------------------------------------------------

\section{Background}

Credit scoring is the set of modelling and decision techniques associated with autonomously adjudicating whether or not a potential borrower should be granted credit \parencite{PerceptronScoring}. At the heart of these systems are loan default prediction models. The techniques involved are used to drive the strategies for determining the amount of credit a borrower should receive, the period of repayment, and the interest rate due on the amount borrowed \parencite{CreditRiskSummary}. \\

Credit scoring systems range in scale from the rating of countries and global international companies to rating personal credits. The systems measure a potential borrower's ability to repay a financial obligation. The systems do not forecast loan profitability. Rather, they are used to reduce credit risk and limit the number of loans that are not repaid, which in turn increases profitability \parencite{EarlyNNScoring}. \newpage

Traditional credit scoring involves considering a borrower's previous loan history when determining their credit score. Loan history data can comprise of loans that were taken from the institution granting the credit, or can be acquired from external credit bureaus. A consumer's previous loan performance directly influences the credit score they receive. If a consumer did not fully repay a previous loan, credit scoring systems will take this into consideration and assign the consumer a lower score \parencite{DynamicBehaviouralScoring}. \\

Since the beginning of the big data era, financial institutions have been able to access rapidly increasing volumes of data. Beyond the volume of data, financial intuitions have been able to access various types of data, from varying sources. Companies have been able to extract data and create features from; clients' short message service history, their call history, and data from clients' social media platforms. Data acquired from these sources is referred to as alternative data.\\

Credit scoring is one of the oldest applications of data analytics \parencite{IntroToCreditModelling}. Prior to the rise of machine learning, more traditional modelling techniques such as logistic regression, linear discriminant analysis and naive Bayes classifiers were used to drive credit scoring systems. \\

Since the rise of big data and machine learning, financial institutions have been able to train and use non-parametric models such as decision trees, support vector machines and neural networks to drive their credit decision systems \parencite{IntroToCreditModelling}. \\

The use of modern machine learning algorithms within credit scoring systems has not been fully supported within the financial industry. Machine learning algorithms are often complex and the predictions they produce can be difficult to explain.  \\

Beyond their complexity, modern machine learning models dynamically evolve and are required to be regularly retrained on different data flows. Tracing the evolution of machine learning models and the data flows they are retrained on poses major issues for financial regulators \parencite{Regulation}. \\

%---------------------------------------------------------------------------------------
%	SECTION 3
%---------------------------------------------------------------------------------------

\section{Aims of Research}

Despite recent advancements in the use of alternative data in credit granting models, the issue is still widely felt \parencite{BigDataMicroFiance}. Research into a solution is limited and has been mainly focused on using contact data on potential borrowers' cellular device and network analysis techniques. \\

This minor dissertation (m.d.) - using data provided by a Nigerian micro-finance company and the Nigerian credit bureaus - seeks to address the following aims:  

\begin{itemize}
    \item Assess if augmenting sociodemographic and credit bureau data with the alternative features used in this project improves the overall performance of loan default prediction models.
    \item Determine if the alternative features used through this dissertation can be used to train accurate loan default prediction models. 
    \item Identify the optimal technique for developing loan default prediction models out of logistic regression, random forests, extreme gradient boosting, and a multi-perceptron neural networks. 
\end{itemize}
    

%---------------------------------------------------------------------------------------
%	SECTION 4
%---------------------------------------------------------------------------------------

\section{Scope of Project}

Only first time loan applicants that had existing Nigerian credit bureau data were considered for this project. The applicants were required to have a credit history in order to measure the impact of augmenting sociodemographic and credit bureau data with alternative data. The models developed only using alternative feature aim to provide an indication as to how well the models would perform on the unbanked population. However, the impacts on financial inclusion are not measured. \\

The regulatory and financial implications of each data category and modelling technique used throughout this project are not investigated. The project only investigates the modelling performance of each technique across the various datasets containing all combinations of the various data categories. \\

The alternative features used throughout this project are generated generated from Short Messaging Service (SMS) data (only messages received from banking intuitions) and Android application data on borrowers' devices. Contact data or other data types on borrowers' devices are not considered. \\

%---------------------------------------------------------------------------------------
%	SECTION 5
%---------------------------------------------------------------------------------------

\section{Layout of the Paper}

Chapter \ref{Chapter2} presents a review of literature related to assigning first-time loan customers with a credit score and how a credit score relates to loan default prediction. Particular attention is given to the modelling techniques that have been used to drive credit scoring, and how those techniques have developed over time. Finally the chapter explores the various alternative data sources used in credit scoring and the techniques applied to make use of the sources. \\

First, Chapter \ref{Chapter3} details the data sources used throughout this project. Secondly, it summarises the data wrangling and feature engineering processes completed to generate the alternative features created during this m.d. Chapter \ref{Chapter3} then details the preprocessing techniques used to created handle missing values, scale the variables, and handle outliers contained within the data used to train and test the models developed throughout this project. \\

Chapter \ref{Chapter4} summarises the various datasets - and the data categories contained in each dataset - that are used to train the loan default prediction models of this project. The chapter further details the feature selection process for each dataset. Then, the various techniques used to train the loan default prediction models are detailed, which includes the hyper-parameters that are tuned for each technique. Finally, the measures used to test whether the alternative features improved model performance and the test used to compare modelling techniques are detailed.  \newpage

The results chapter of this project - Chapter \ref{Chapter5} - displays the results of the feature selection process completed for each dataset. The chapter then displays the performance of each modelling technique across each dataset. Finally, the chapter summarises the findings of the research. \\

The concluding chapter of this project - Chapter \ref{Chapter6} - presents the findings and conclusions of the project and indicates how the research conducted could be furthered.  


